\documentclass[specialist,
               substylefile = spbu.rtx,
               subf,href,colorlinks=true, 12pt]{disser}

\usepackage[a4paper,
            mag=1000, includefoot,
            left=3cm, right=1.5cm, top=2cm, bottom=2cm, headsep=1cm, footskip=1cm]{geometry}
\usepackage[T2A]{fontenc}
\usepackage[utf8]{inputenc}
\usepackage[english,russian]{babel}
\ifpdf\usepackage{epstopdf}\fi


\usepackage{ucs}
\usepackage{cite}
\usepackage{amsmath}
\usepackage{amsthm}
\usepackage{amsfonts}
\usepackage{amssymb}
\usepackage{nicefrac}
\usepackage{csvsimple}

\usepackage{graphicx}


% Использовать полужирное начертание для векторов
\let\vec=\mathbf

% Включать подсекции в оглавление
\setcounter{tocdepth}{2}

\graphicspath{{fig/}}


\theoremstyle{definition}
\newtheorem{definition}{Определение}

\newtheorem{theorem}{Теорема}

% Expectation symbol
\DeclareMathOperator*{\E}{\mathrm{E}}
\DeclareMathOperator*{\D}{\mathrm{D}}
\DeclareMathOperator*{\sign}{sign}
\DeclareMathOperator*{\argmin}{arg\,min}
\DeclareMathOperator*{\Int}{Int}
\DeclareMathOperator*{\err}{err}


\newcommand\norm[1]{\left\lVert#1\right\rVert}
\newcommand\abs[1]{\left\lvert#1\right\rvert}


%----------------------------------------------------------------
\begin{document}


\institution{%
    Санкт-Петербургский государственный университет \\
    Прикладная математика и информатика \\
    Вычислительная стохастика и статистические модели
}

\title{Отчет о научно-исследовательской работе}

% Тема
\topic{\normalfont\scshape%
    Оптимальные планы для оценивания производных в полиномиальной регрессионной модели без свободного члена}

% Автор
\author{Барсуков Егор Вячеславович}

% Научный руководитель
\sa       {В.\,Б.~Мелас}
\sastatus {д.\,ф.-м.\,н., профессор}

% Рецензент
%\rev      {П.\,П.~Петров}
%\revstatus{к.\,ф.-м.\,н., доцент}

% Город и год
\city{Санкт-Петербург}
\date{2019}

\maketitle

\tableofcontents

\addcontentsline{toc}{chapter}{Введение}
\chapter*{Введение}

  Рассмотрим регрессионную модель

  \begin{equation}
  \label{eq:regres}
    y_j = \theta^\top f_j(x_j) + \varepsilon_j, \quad j = 1 \ldots N, \, x_j \in \mathcal{X},
  \end{equation}
  где $N$ --- количество экспериментов, $\mathcal{X} \subset \mathbb{R}$, $f(x) = \left(f_1(x), \ldots, f_n(x) \right)^\top$ --- регрессионная функция, $\theta = \left( \theta_1, \ldots, \theta_n \right)^\top$ --- неизвестные параметры, $\varepsilon_i$ ---  некоррелированные ошибки наблюдения. При этом $\E [\varepsilon_i] = 0$, $\D [\varepsilon_i] = \sigma^2$.
  
  Для улучшения в каком-либо смысле некоторых оценок по данной модели при минимизации $N$  строят \textit{планы эксперимента}, т.е. наборы точек $x_i \in \mathcal{X}$ в каждой из которых должно быть произведено $n_i$ экспериментов так, что $\sum^N_i n_i = N$. В каком именно смысле будет улучшена оценка зависит от одного из множества критериев оптимальности. В этой работе будут рассматриваться $c$-оптимальные планы эксперимента, определение которым будет дано далее.
  
  В работе будут рассматриваться полиномиальные регрессионные модели, т.е. $f_i(x) = x^{k_i}$. Такие модели были хорошо изучены и для многих случаев были построены явные аналитические решения. Множество работ были посвящены нахождению $\mathrm{D}$-оптимальных планов \cite{hoel1958, studden1980, dette1990, dette2001}. Также существуют для такой модели явные решения для нахождения $\mathrm{E}$-оптимальных планов \cite{pukelsheim1993, dette1993, heiligers1994, dette1993_2}.
  
  С-оптимальным планом эксперимента является план минимизирующий дисперсию значения скалярного произведения $\theta$ и $c$ для заданного $c \in \mathbb{R}^n$. В общем случае нахождение $c$-оптимальных планов может быть достаточно сложно. Некоторые решения с очень маленьким количеством неизвестных параметров описаны опираясь на использование теоремы Элвинга \cite{elfving1952}. 
  
  Часто рассматриваются задача нахождения оптимального плана при $c = f(z)$ для произвольного $z$ --- задача экстраполяции и $c = f'(z)$ --- задача оценки производной, они были решены в общем виде для обычной полиномиальной регрессии вида $f(x) = (1, x, \ldots, x^n)^\top$ \cite{hoel1964, melas2010}. 
  
  В этой работе рассмотрен случай нахождения нахождения плана для оценивания производной при полиномиальной модели без свободного члена при $\mathcal{X} = [0, d]$. Этот случай существенно отличается от $\mathcal{X} = [-1, 1]$, который был рассмотрен в \cite{melasmain}. Также был значительно улучшен алгоритм численного нахождения $c$-оптимальных планов эксперимента для произвольного $c$.
  
  

\chapter{Общие сведения}

\section{Определения}

Согласно \cite{kiefer1974} непрерывным планом эксперимента в регрессионной модели (\ref{eq:regres}) будем называть дискретную вероятностную меру
  \begin{equation*}
    \xi = 
      \begin{pmatrix}
        x_1 & \ldots & x_m \\
        \omega_1 & \ldots & \omega_m
      \end{pmatrix}, \quad x_i \in \mathcal{X}.
  \end{equation*}
  Для проведения $N$ измерений с таким планом эксперимента необходимо провести $n_i \approx N \omega_i$ измерений в точке $x_i$ таким образом, чтобы $\sum^m_i n_i = N$.

  Для непрерывного плана эксперимента определим информационную матрицу следующим образом
  \begin{equation*}
    M(\xi) = \int_{\mathcal{X}} f(x) f^\top(x) \xi (dt).
  \end{equation*}
  
  \begin{definition}
  \label{def:coptim}
  \textit{$c$-оптимальным планом} для некоторого вектора $c$ называется план эксперимента $\xi$ минимизирующий следующую функцию
  \begin{equation*}
    \Phi(\xi) = \begin{cases}
      c^\top M(\xi)^{-} c, \quad \text{если существует } v \text{, такой, что } \; c = M(\xi) v\\
      +\infty, \quad  \text{иначе}
    \end{cases},
  \end{equation*}
  где $M(\xi)^{-}$ --- матрица, обобщенно обратная к информационной матрице плана $\xi$.
  План называется допустимым, если существует такое $v$, что $c = M(\xi) v$.
  \end{definition}
  
  Как было отмечено в введении, $c$-оптимальный план минимизирует дисперсию несмещенной МНК оценки $c^\top \hat{\theta}$ линейной комбинации $c^\top \theta$ \cite{dette1993_2}.
  
  \begin{definition}
  Если $c = f(z)$ для некоторого $z \in \mathbb{R}$, то соответствующий $c$-оптимальный план называется \textit{оптимальном планом экстраполяции} в точке $z$.
  \end{definition}
  
  \begin{definition}
  Если $c = f'(z)$ для некоторого $z \in \mathbb{R}$, то соответствующий $c$-оптимальный план называется \textit{оптимальным планом для оценки производной} в точке $z$.
  \end{definition}
  
  \section{Теорема Элвинга}
  
  Для решения задачи нахождения $\mathrm{c}$-оптимальных планов в множестве случаев (в том числе в данной работе) используется теорема Элвинга, являющаяся геометрически интерпретируемым критерием $\mathrm{c}$-оптимальности плана эксперимента.
  \begin{theorem}
  \label{th:elfving}
  (Элвинга) \cite{elfving1952}
  Допустимый план $\xi^\star$ с носителем $x_1, \ldots, x_m \in \mathcal{X}$ и весами $\omega_1, \ldots, \omega_m$ является $c$-оптимальным тогда и только тогда, когда существует $p \in \mathbb{R}^k$ и константа $h$ такие, что выполняются следующие условия:
  \begin{subequations}
  \label{eq:elfving}
  \begin{align}
	&\abs{p^\top f(x_i)} = 1 &&i=1..m \leqslant n \label{eq:elfving:eq1} \\
	&\abs{p^\top f(x)} \leqslant 1  &&x \in \mathcal{X} \label{eq:elfving:eq2} \\
	&c = h \sum_{i=1}^m \omega_i f(x_i) p^\top f(x_i) \label{eq:elfving:eq3}.
  \end{align}
  Кроме того
  \begin{equation*}
  	h^2 = c^\top M^{-}(\xi^{*})c
  \end{equation*}
  \end{subequations}
  \end{theorem}
	Функция $p^\top f(x_i)$ в определениях теоремы Элвинга называют \textit{экстремальным многочленом}.
	
	\section{Явная формула для весов оптимального плана}
	
	\begin{theorem}
	\label{th:weights}
	Оптимальный план для оценивания производной полиномиальной модели без свободного члена с опорными точками $t_1^*, \ldots, t_m^*$, где $m=n$ или $m=n-1$ имеет веса вычисленные по следующей формуле:	
	
	\begin{equation}
		\omega_i = \frac{\abs{L'_i(z)}}{\sum_{j=1}^m \abs{L'_j(z)}},
	\end{equation}
	где $L_i$ задается следующим образом
	\begin{equation*}
		L_{i}(x) = \frac{x \prod_{l=1}^n (x - t_l^*)}{t_i^* \prod_{l \neq i}^n (t_i^* - t_l^*)},
	\end{equation*}	
	то есть является $i$-ым базисным многочленом Лагранжа без нулевого члена построенным по точкам $t_1^*, \ldots, t_m^*$.
	\end{theorem}
	
	
	
  
  \chapter{План для нахождения производной на промежутке $[0, d]$}


	Построим полином без свободного члена $S_n(x)$ степени $n$, не превосходящий по модули единицу на промежутке $[0, 1]$, и достигающий её в $n$ точках. Пусть $T_n(x)$ --- многочлен Чебышёва степени $n$. Тогда по свойствам многочленов Чебышёва $T_n$ не превосходит по модулю единицу на промежутке $[-1, 1]$ и достигает её в $n+1$ точках, в том числе в точках -1 и 1. Известно, что корни $T_n$ имеют следующий вид
	
	\begin{equation*}
		x_{i}=\cos \left({\frac {\pi (i+1/2)}{n}}\right),\quad i=0,\ldots ,n-1.
	\end{equation*}	 
	
	Если мы возьмём самый маленький корень $x_{\text{min}} = - \cos \frac{\pi}{2n}$ и положим $\widehat{S_n}(x) = T_n(x + x_{\text{min}})$, то $\widehat{S_n}$ будет являться полиномом степени $n$ с нулевым свободным членом, так как $\widehat{S_n}(0) = 0$ по построению. При этом $\abs{\widehat{S_n}(x)} \leqslant 1$ для $x$ на промежутке $[0, 1+ \cos \left(\frac{\pi}{2n}\right)]$, при этом в этом промежутке абсолютная величина достигает единицы $n$ раз в силу того, что левый край не равен 1.
	
	Для того, чтобы привести промежуток к виду $[0, 1]$ достаточно добавить множитель $1 + \cos \frac{\pi}{2n}$  к $x$ в левой части определения $\widehat{S_n}$. После этого получается многочлен, удовлетворяющий всем требуемым свойствам
	
	\begin{equation*}
		S_n(x) = T_n \left(x \left(1 + \cos \frac{\pi}{2n} \right) - \cos \frac{\pi}{2n} \right).
	\end{equation*}	
	
	Для более общего случая в виде промежутка $[0, d]$ требуемый многочлен (обозначим его $S_n^d$) можно выразить из $S_n(x)$ как $S_n^d(x) = S_n\left(\frac{x}{d}\right)$.
	
	Исходя из данного построения и известных экстремальных точек многочлена Чебышёва, можно легко выразить экстремальные точки $S_n$. Обозначим их как $s_{i, n}$, тогда
	
	\begin{equation*}
		s_{i, n} = \frac{\cos \frac{(n - i) \pi}{n} + cos \frac{\pi}{2n}}{1 + \cos \frac{\pi}{2n}} , \, \quad i = 1, \ldots, n 
	\end{equation*}
	при этом
	\begin{equation*}
		0 < s_{1, n} < \ldots < s_{n, n}
	\end{equation*}
	и
	\begin{equation*}
		S_n(s_{i, n}) = (-1)^{n + i} , \, \quad i = 1, \ldots, n 
	\end{equation*}

	Построим базисные полиномы Лагранжа степени $n$ без нулевого члена по точкам $\{s_{i, n}\}_{i=1}^n$
	\begin{equation*}
		L_{i}(x) = \frac{x \prod_{l=1}^n (x - s_{l,n})}{s_{i, n} \prod_{l \neq i}^n (s_{i,n} - s_{l,n})}
	\end{equation*}	
	
	Так как теорема 4 о весах оптимального плана в случае полиномиальной модели работает для любых промежутков, то для нахождения весов можно использовать её.
	
	\begin{equation*}
		\omega_i = \frac{\abs{L_i'(z)}}{\sum_{j=1}^n \abs{L_j'(z)}}
	\end{equation*}
	
	В силу свойств многочленов Чебышёва $S_n(x)$ и $-S_n(x)$ --- единственные многочлены степени $n$ без нулевого члена, которые удовлетворяют свойствам 1-2 теоремы Элвинга,поэтому осталось проверить для только свойство 3. Для этого введем обозначения $F = \left(s_{j,n}^i\right)^n_{i, j = 1}$, $h = \sum_{j=1}^n \abs{L_j'(z)}$ и $\beta = \left( \abs{L'_i(z)} (-1)^{i+n} \right)_{i=1}^n$. Так как $s_{j, n}$ при $i = 1, \ldots , n$ являются экстремальными точками многочлена $S_n$ и при этом $S_n(s_{j, n}) = (-1)^{j+n}$, то выполнение равенства 
	\begin{equation}
		\label{th:mateq3}
		f'(z) = hF\beta
	\end{equation}
	 при $\omega_i \geqslant 0 , \, i = 1, \ldots, n$ и $ \sum_{i=1}^n \omega_i = 1$ эквивалентно выполнению условия 3 теоремы Элвинга для нахождения оптимального плана оценки производной в точке $z$.
	 
	Так как равенство $F^{-1}F = I_n$, где $I_n$ --- единичная матрица размера $n$, можно переписать, как систему равенств
	\begin{equation}
		\label{eq:toleg}
		e_i^{\top} F^{-1} f(s_{j,n}) = \delta_{ij} \, , \quad i, j = 1, \ldots , n ,
	\end{equation}
	где $\delta_{ij}$ --- дельта Кронекера, а $e_i$ --- $i$--ый единичный вектор. Поскольку в левой части равенств (\ref{eq:toleg}) содержатся многочлены без нулевого коэффициента степени не больше $n$ вычисленные в точках $s_{j, n}$, $j=1, \ldots , n$, а для каждого $i$ существует только одно $j$, такое, что $\delta_{ij} \neq 0$, то они определяют все базисные многочлены Лагранжа без нулевого члена степени $n$ вычисленные в точках $s_{j, n}$, $j=1, \ldots , n$, таким образом
	\begin{equation}
		\label{eq:leg}
		e_i^{\top} F^{-1} f(z) = L_i(z) \, , \quad i, j = 1, \ldots , n .
	\end{equation}
	Если в предыдущем выражении вычислить производную по $z$ и переписать полученное выражение в векторной форме получим
	\begin{equation}
		\label{eq:vecleg}
		f'(z) = F \left( L'_1(z), \ldots, L'_n(z) \right)^\top.
	\end{equation}
	Приравняв правые части (\ref{eq:vecleg}) и (\ref{th:mateq3}) и домножив равенство на $F^{-1}$ слева, получаем, что
	\begin{equation}
		\label{eq:vecleg}
		h \beta = \left( L'_1(z), \ldots, L'_n(z) \right)^\top,
	\end{equation}
	что с учетом введенных ранее обозначений влечет, что $\sign (L'_i(z)) = \sign((-1)^{i+n}) $, $i = 1, \ldots, n$ или, вспомнив, что экстремальным многочленом также может быть $-S(x)$,  $\sign (L'_i(z)) = \sign((-1)^{i+n+1}) $, $i = 1, \ldots, n$.
	
	Таким образом для того, чтобы доказать, что оптимальный план находится в точках $(s_{i, n})_{i=1}^n$ , $i = 1, \ldots, n$ с указными ранее весами, осталось доказать равенство знаков $L'_i(z)$ и $\pm S_n(s_{i, n})$. Но так как знаки экстремальных точек многочлена $S_n$ чередуются, достаточно показать при каких $z$ выражения $(-1)^i L'_i(z)$ для имеет одинаковый знак для $i=1,\ldots , n$.
	
	Обозначим корни многочлена $L'_i$ как $u_{i,1}, \ldots, u_{i,n-1}$, $i = 1, \ldots, n$. Так как для $L_i$ и $L_j$ выполняются требования леммы 2 для любых $i$ и $j$ таких что $i < j$, то последовательно применяя ее для всех базисных многочленов получаем, что
	\begin{equation}
	\label{eq:droots}
		u_{n, 1} < u_{n-1, 1} < \ldots < u_{1, 1} < u_{n, 2} < u_{n-1, 2} < \ldots < u_{1, 2} < \ldots < u_{1, n-1}.
	\end{equation}
	
	Можно видеть, что, так как все узловые точки больше нуля, знак многочлена $L_i(z)$ при $z \to -\infty$ будет равен $(-1)^{n+i+1}$. В то же время знак $L'_i(z)$ будет противоположным $L_i(z)$ так как меняется четность многочлена и при этом не меняется знак при старшем коэффициенте, то есть $\sign(L'_i(z)) = \sign((-1)^{n+i})$ при $z \to -\infty$. И, следовательно, $\sign((-1)^i L'_i(z)) = \sign((-1)^{n+2i}) = \sign((-1)^{n})$ при $z \to -\infty$, то есть $\sign((-1)^i L'_i(z))$ не зависит от $i$ и имеет постоянный знак для любых $i$, что означает, что при $z \in (-\infty, u_{n, 1})$ третье условие теоремы Элвинга выполняется и план является оптимальным.
	
	Осталось изучить как ведут себя знаки $\sign((-1)^i L'_i(z))$ на остальных промежутках. На промежутках $[u_{j, 1}, u_{j, n-1}]$ каждый базисный многочлен меняет свой знак ровно 1 раз и на этих промежутках знаки производных не совпадают со знаками экстремального многочлена, а на промежутках $(u_{j, n-1}, u_{j-1, 1})$ нет ни одного корня и поэтому $\sign((-1)^i L'_i(z)) = \sign ((-1)^{n + j}$ при $z \in (u_{j, n-1}, u_{j-1, 1})$, что также подтверждает третье условие теоремы Элвинга и показывает, что показанный план оптимален для $j=1, \ldots n-1$.
	
	На промежутке $(-\infty, u_{1, n-1}$ каждый базисный многочлен поменял свой знак одинаковое количество раз, а так как при $z \to -\infty$ условие выполнялось, то при $z \in (u_{1, n-1}, +\infty)$ план также является оптимальным.
	
	
	\chapter{Численное нахождение оптимальных планов}
	 
	
	
	
	
	
	

	
	%\nocite{*}
	%\bibliographystyle{gost2008}
	%\bibliography{report}
	
\end{document}