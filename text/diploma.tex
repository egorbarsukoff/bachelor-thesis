\documentclass[specialist,
               substylefile = spbu.rtx,
               subf,href,colorlinks=true, 12pt]{disser}

\usepackage[a4paper,
            mag=1000, includefoot,
            left=3cm, right=1.5cm, top=2cm, bottom=2cm, headsep=1cm, footskip=1cm]{geometry}
\usepackage[T2A]{fontenc}
\usepackage[utf8]{inputenc}
\usepackage[english,russian]{babel}
\ifpdf\usepackage{epstopdf}\fi


\usepackage{ucs}
\usepackage{cite}
\usepackage{amsmath}
\usepackage{amsthm}
\usepackage{amsfonts}
\usepackage{amssymb}
\usepackage{nicefrac}
\usepackage{csvsimple}

\usepackage{graphicx}


% Использовать полужирное начертание для векторов
\let\vec=\mathbf

% Включать подсекции в оглавление
\setcounter{tocdepth}{2}

\graphicspath{{fig/}}


\theoremstyle{definition}
\newtheorem{definition}{Определение}

\newtheorem{theorem}{Теорема}
\newtheorem{lemma}{Лемма}

% Expectation symbol
\DeclareMathOperator*{\E}{\mathrm{E}}
\DeclareMathOperator*{\D}{\mathrm{D}}
\DeclareMathOperator*{\sign}{sign}
\DeclareMathOperator*{\argmin}{arg\,min}
\DeclareMathOperator*{\Int}{Int}
\DeclareMathOperator*{\err}{err}


\newcommand\norm[1]{\left\lVert#1\right\rVert}
\newcommand\abs[1]{\left\lvert#1\right\rvert}


%----------------------------------------------------------------
\begin{document}


\institution{%
    Санкт-Петербургский государственный университет \\
    Прикладная математика и информатика \\
    Вычислительная стохастика и статистические модели
}

\title{Отчет о научно-исследовательской работе}

% Тема
\topic{\normalfont\scshape%
    Оптимальные планы для оценивания производных в полиномиальной регрессионной модели без свободного члена}

% Автор
\author{Барсуков Егор Вячеславович}

% Научный руководитель
\sa       {В.\,Б.~Мелас}
\sastatus {д.\,ф.-м.\,н., профессор}

% Рецензент
%\rev      {П.\,П.~Петров}
%\revstatus{к.\,ф.-м.\,н., доцент}

% Город и год
\city{Санкт-Петербург}
\date{2019}

\maketitle

\tableofcontents

\addcontentsline{toc}{chapter}{Введение}
\chapter*{Введение}

  Рассмотрим регрессионную модель

  \begin{equation}
  \label{eq:regres}
    y_j = \theta^\top f_j(x_j) + \varepsilon_j, \quad j = 1 \ldots N, \, x_j \in \mathcal{X},
  \end{equation}
  где $N$ --- количество экспериментов, $\mathcal{X} \subset \mathbb{R}$, $f(x) = \left(f_1(x), \ldots, f_n(x) \right)^\top$ --- регрессионная функция, $\theta = \left( \theta_1, \ldots, \theta_n \right)^\top$ --- неизвестные параметры, $\varepsilon_i$ ---  некоррелированные ошибки наблюдения. При этом $\E [\varepsilon_i] = 0$, $\D [\varepsilon_i] = \sigma^2$.
  
  Для того, чтобы при фиксированном количестве наблюдений $N$ получить наилучшую в каком-либо смысле оценку $\theta$ строят \textit{планы эксперимента} т.е. наборы точек $x_i \in \mathcal{X}$ в каждой из которых должно быть произведено $n_i$ экспериментов так, что $\sum^N_i n_i = N$. В каком именно смысле будет улучшена оценка параметров модели зависит от выбора критерия одного из нескольких критериев оптимальности.
  
  В работе будут рассматриваться полиномиальные регрессионные модели, т.е. $f_i(x) = x^{k_i}$. Для таких моделей во многих случаях явным образом были описаны оптимальные планы. Несколько работ были посвящены нахождению $\mathrm{D}$-оптимальных планов \cite{hoel1958, studden1980, dette1990, dette2001}. Также существуют для такой модели явные решения для нахождения $\mathrm{E}$-оптимальных планов \cite{pukelsheim1993, dette1993, heiligers1994, dette1993_2}.
  
  В этой работе будут рассматриваться $c$-оптимальные планы эксперимента. Ими являются планы минимизирующие дисперсию значения скалярного произведения $\theta$ и $c$ для заданного $c \in \mathbb{R}^n$ \cite{dette1993_2}. В общем случае нахождение $c$-оптимальных планов может быть достаточно сложно: для случаев малой размерности решение можно найти используя теорему Элвинга \cite{elfving1952}, однако явного решения для произвольного $c$ не существует.
  
  В практических приложениях важны несколько частных случаев: $c = f(z)$ для некоторого $z \notin \mathcal{X}$ --- задача экстраполяции в точке $z$ и $c = f'(z)$ --- задача оценки производной в точке $z$. Для обычной полиномиальной модели оптимальный план экстраполяции был описан достаточно давно \cite{hoel1964}, также существует несколько явных решений для задачи оценки производной в некоторых случаях \cite{melas2010, melasmain}.
  
  В этой работе рассмотрен случай нахождения нахождения планов для оценки производной при полиномиальной модели без свободного члена при $\mathcal{X} = [0, d]$. Такая модель, например, может быть использована в тех случаях, когда исходя из практической задачи значения $z$ могут принимать только положительные значения и существует априорное знание о значении функции в точке 0. Простым примером такой функции является зависимости расстояния до начальной точки от времени.
  
  
  

\chapter{C-оптимальные планы эксперимента}

\section{Определения}

  \begin{definition}
  Согласно \cite{kiefer1974} непрерывным планом в регрессионной модели \eqref{eq:regres} будем называть дискретную вероятностную меру
  \begin{equation*}
    \xi = 
      \begin{pmatrix}
        x_1 & \ldots & x_m \\
        \omega_1 & \ldots & \omega_m
      \end{pmatrix}, \quad x_i \in \mathcal{X},
  \end{equation*}
   где $\omega_i \geqslant 0, \, \sum_i^n \omega_i = 1$, $i = 1, \ldots, m$.
  \end{definition}

  Для проведения $N$ измерений с мерой $\xi$ необходимо провести $n_i \approx N \omega_i$ измерений в точке $x_i$ таким образом, чтобы $\sum^m_i n_i = N$.

  \begin{definition}
  Информационной матрицей для непрерывного плана эксперимента заданного мерой $\xi$ является 
  \begin{equation*}
    M(\xi) = \int_{\mathcal{X}} f(x) f^\top(x) \xi (dt).
  \end{equation*}
  \end{definition}
  
  \begin{definition}
  \label{def:coptim}
  \textit{$c$-оптимальным планом} для некоторого вектора $c$ называется план эксперимента $\xi$ минимизирующий следующую функцию
  \begin{equation*}
    \Phi(\xi) = \begin{cases}
      c^\top M(\xi)^{-} c, \quad \text{если существует } v \text{, такой, что } \; c = M(\xi) v\\
      +\infty, \quad  \text{иначе}
    \end{cases},
  \end{equation*}
  где $M(\xi)^{-}$ --- матрица, обобщенно обратная к информационной матрице плана $\xi$.
  План называется допустимым, если существует такое $v$, что $c = M(\xi) v$.
  \end{definition}
  
  Как было отмечено во введении, $c$-оптимальный план минимизирует дисперсию несмещенной оценки по методу наименьших квадратов $c^\top \hat{\theta}$ линейной комбинации $c^\top \theta$ \cite{dette1993_2}.
  
  \begin{definition}
  Если $c = f(z)$ для некоторого $z \in \mathbb{R}$, то соответствующий $c$-оптимальный план называется \textit{оптимальном планом экстраполяции} в точке $z$.
  \end{definition}
  
  \begin{definition}
  Если $c = f'(z)$ для некоторого $z \in \mathbb{R}$, то соответствующий $c$-оптимальный план называется \textit{оптимальным планом для оценки производной} в точке $z$.
  \end{definition}
  
  \section{Теорема Элвинга}
  
  Для решения задачи нахождения $c$-оптимальных планов во множестве случаев (в том числе в данной работе) используется теорема Элвинга, являющаяся геометрически интерпретируемым критерием $c$-оптимальности плана эксперимента.
  \begin{theorem}
  \label{th:elfving}
  (Элвинга) \cite{melas2010}
  Допустимый план $\xi^\star$ с носителем $x_1, \ldots, x_m \in \mathcal{X}$ и весами $\omega_1, \ldots, \omega_m$ является $c$-оптимальным тогда и только тогда, когда существует $p \in \mathbb{R}^k$ и константа $h$ такие, что выполняются следующие условия:
  \begin{subequations}
  \label{eq:elfving}
  \begin{align}
	&\abs{p^\top f(x_i)} = 1 &&i=1..m \leqslant n \label{eq:elfving:eq1} \\
	&\abs{p^\top f(x)} \leqslant 1  &&x \in \mathcal{X} \label{eq:elfving:eq2} \\
	&c = h \sum_{i=1}^m \omega_i f(x_i) p^\top f(x_i) \label{eq:elfving:eq3}.
  \end{align}
  Кроме того
  \begin{equation*}
  	h^2 = c^\top M^{-}(\xi^{*})c
  \end{equation*}
  \end{subequations}
  \end{theorem}
	Функцию $p^\top f(x_i)$ в определениях теоремы Элвинга в этой работе также будет называться \textit{экстремальным многочленом}.
	
	\section{Явная формула для весов оптимального плана}
	
	\begin{theorem}
	\label{th:weights}
	Оптимальный план для оценивания производной полиномиальной модели без свободного члена с опорными точками $t_1^*, \ldots, t_m^*$, где $m=n$ или $m=n-1$ имеет веса вычисленные по следующей формуле:	
	
	\begin{equation}
	\label{eq:weights}
		\omega_i = \frac{\abs{L'_i(z)}}{\sum_{j=1}^m \abs{L'_j(z)}},
	\end{equation}
	где $L_i$ задается следующим образом
	\begin{equation}
		\label{eq:lagr}
		L_{i}(x) = \frac{x \prod_{l=1}^n (x - t_l^*)}{t_i^* \prod_{l \neq i}^n (t_i^* - t_l^*)},
	\end{equation}	
	то есть является $i$-ым базисным многочленом Лагранжа без нулевого члена построенным по точкам $t_1^*, \ldots, t_m^*$.
	\end{theorem}
	
	
	%\chapter{Численное нахождение оптимальных планов}
	
	%Из определения $c$-оптимальных планов следует, что такой план всегда должен существовать, однако в общем случае не существует явного решения
	
	%Ранее мной был показан алгоритм численного нахождения $c$-оптимальных планов эксперимента для произвольных $c$, моделей, и промежутков, однако, основываясь на минимизации достаточно сложной функции построенной из условий теоремы Элвинга и процесс нахождения был весьма долгим и сходился не всегда. В этой главе описан новый метод нахождения оптимальных планов который сходится существенно быстрее и для всех начальных данных.
	
	
	
  
	\chapter{План для нахождения производной на промежутке с началом в нуле}
  
  Здесь будут описаны оптимальные планы для нахождения производной на промежутке вида $[0, d]$ в модели без нулевого члена. Для промежутков вида $[-1, 1] $ такие планы были описаны в \cite{melasmain}, однако в таком случае полученные решения значительно отличаются от полученных в этом разделе. Здесь, не умаляя общности, будет доказана теорема для случая $[0, 1]$, а в конце главы будет показано, как он переносится в общий вид. 
  
  Везде в этой главе считается, что $f(x) = (x, x^2, \ldots, x^n)^\top$, то есть рассматривается модель без нулеого члена.
  
  \section{План для нахождения производной на промежутке $[0, 1]$}
  
  При доказательстве основной теоремы этой главы будет использована следующая лемма.
  
  \begin{lemma}
  \label{lemma:droots}
  	Пусть $P_1(x)$ и $P_2(x)$ --- многочлены степени $n$ с корнями $t^1_1 < \ldots < t^1_n$ и $t^2_1 < \ldots < t^2_n$ соответственно. При этом корни располагаются следующим образом:
  	\begin{equation*}
  		t^1_1 \leqslant t^2_1 < t^1_2 \leqslant t^2_2 < \ldots < t^1_n \leqslant t^2_n,
  	\end{equation*}
  	где хотя бы одно из неравенств $t_l^1 \leqslant t_l^2$ ($l = 1, \ldots, n$) является строгим. Также обозначим корни многочленов $P'_1(x)$ и $P'_2(x)$ как $v_1^1, \ldots, v_{n-1}^1$ и $v_1^2, \ldots, v_{n-1}^2$. Тогда справедливо следующее выражение
  	\begin{equation*}
  		v^1_1 < v^2_1 < v^1_2 < v^2_2 < \ldots < v^1_n < v^2_n.
  	\end{equation*}
  \end{lemma}
  Доказательство этого утверждения можно найти в \cite{sahmphd} или в приложении к \cite{melasmain}.
	
	\begin{theorem}
	\label{th1}
		Пусть для $i = 1, \ldots, n$ 
		\begin{equation}
			\label{eq:th1:points}
			x_i^* = \frac{\cos \frac{(n - i) \pi}{n} + cos \frac{\pi}{2n}}{1 + \cos \frac{\pi}{2n}} ,
		\end{equation}
		и корни производной многочлена $L_i$ из формулы~\eqref{eq:lagr} построенного по точкам $\{x_j^*\}_j^n$ равны $u_1^i, \ldots, u_{n-1}^i$, причем $u_k^i \leqslant u_l^i$ при $k < l$. Тогда при $z \in \left(-\infty, u_1^1) \cup (u_{n-1}^n, +\infty \right)$ или $z \in \left( u_{n-1}^{j+1}, u_1^{j} \right)$ для $j=1, \ldots, n-1$ оптимальный план для нахождения производной в точке $z$ в полиномиальной модели степени $n$ без свободного члена на промежутке $[0, 1]$ имеет опорные точки $\{x_j\}_j^n$ и веса вычисленные по формуле \eqref{eq:weights}. Также при $z$ не лежащих в этих промежутках не существует оптимального плана состоящего из $n$ точек.
	\end{theorem}
	\begin{proof}

	Полученное здесь решение основывается на применении варианта теоремы Элвинга описанного на странице \pageref{th:elfving}, которая, в том числе, утверждает, что для любого $c$-оптимального плана должен существовать соответствующий многочлен $p^\top f(x)$, который не превосходит по модулю единицу на соответствующем промежутке и достигает её в опорных точках плана. Поэтому первым шагом будет нахождение таких многочленов степени $n$.
	
	Построим полином без свободного члена $S_n(x)$ степени $n$, не превосходящий по модули единицу на промежутке $[0, 1]$, и достигающий её в $n$ точках. Пусть $T_n(x)$ --- многочлен Чебышёва степени $n$. Тогда по свойствам многочленов Чебышёва $T_n$ не превосходит по модулю единицу на промежутке $[-1, 1]$ и достигает её в $n+1$ точках, в том числе в точках -1 и 1. Известно, что корни $T_n$ имеют следующий вид
	
	\begin{equation*}
		t_{i}=\cos \left({\frac {\pi (i+1/2)}{n}}\right),\quad i=0,\ldots ,n-1.
	\end{equation*}	 
	
	Если мы возьмём самый маленький корень $t_{\text{min}} = - \cos \frac{\pi}{2n}$ и положим $\widehat{S_n}(x) = T_n(x + t_{\text{min}})$, то $\widehat{S_n}$ будет являться полиномом степени $n$ с нулевым свободным членом, так как $\widehat{S_n}(0) = 0$ по построению. При этом $\abs{\widehat{S_n}(x)} \leqslant 1$ для $x$ на промежутке $[0, 1+ \cos \left(\frac{\pi}{2n}\right)]$ и в этом промежутке абсолютная величина достигает единицы $n$ раз в силу того, что левый край не равен 1.
	
	Для того, чтобы привести промежуток к виду $[0, 1]$ достаточно добавить множитель $1 + \cos \frac{\pi}{2n}$  к $x$ в левой части определения $\widehat{S_n}$. После этого получается многочлен, удовлетворяющий всем требуемым свойствам
	
	\begin{equation*}
		S_n(x) = T_n \left(x \left(1 + \cos \frac{\pi}{2n} \right) - \cos \frac{\pi}{2n} \right).
	\end{equation*}	
	
	Экстремальные точки $T_n(x)$ расположены в точках
	\begin{equation*}
		\widehat{x_i^*} = \cos \frac{(n - i)\pi}{n}, \, \quad i = 0, \ldots, n,
	\end{equation*}
	поэтому экстремальные точки $S(x)$ на промежутке $[0, 1]$ расположены в точках
	\begin{equation*}
		x_i^* = \frac{\cos \frac{(n - i) \pi}{n} + \cos \frac{\pi}{2n}}{1 + \cos \frac{\pi}{2n}} , \, \quad i = 1, \ldots, n, 
	\end{equation*}
	при этом важно, что индексы у $x_i^*$ начинаются с 1, в то время как у $\widehat{x_i^*}$ c 0. Так происходит из-за того, что наименьшая экстремальная точка $S_n$ на всем промежутке оказывается меньше нуля и не попадает в требуемый промежуток.
	
	Также нужно отметить, что
	\begin{equation*}
		0 < x_1^* < x_2^* < \ldots < x_n^* , \, \quad i = 1, \ldots, n 
	\end{equation*}
	и
	\begin{equation*}
		S_n(x_i^*) = (-1)^{n + i} , \, \quad i = 1, \ldots, n 
	\end{equation*}
	
	Покажем, что $S_n(x)$ --- это единственный многочлен (с точностью до знака) без нулевого члена, удовлетворяющий свойствам свойствам \eqref{eq:elfving:eq1} и \eqref{eq:elfving:eq2} на промежутке $[0, 1]$. Пусть $P_n(x)$ --- многочлен степени $n$ и при этом $P_n(0) = 0$, $P_n(1) = 1$ и в промежутке $[0, 1)$ этот многочлен имеет ровно $n-1$ экстремум $t_1 < \ldots < t_{n-1}$ в которых он равен $\pm 1$. В силу того, что нам требуется единственность с точностью до знака, тот факт, что $P_n(1)$ был положен положительным, не умаляет общности. Так как $P'_n(x)$ имеет степень $n-1$ и, соответственно, имеет $n-1$ корней, которые были обозначены как $t_1, \ldots, t_{n-1}$, то только в этих точках этот многочлен может менять свою монотонность, но также $\abs{P_n(t_i)} = 1$ для $i = 1, \dots, n-1$, поэтому в этих точках он меняет свою монотонность и $P_n(t_i) = (-1)^{n-1-i}$. Также поэтому при $x \leqslant t_1$ (ранее было определено, что $t_1 > 0$) $\sign(P'_n(x)) = (-1)^n$, а учитывая, что $P(0) = 0$ и $P(t_1) = (-1)^n$, то должна существовать такая единственная точка $t_0 < 0$, что $P(t_0) = (-1)^{n+1}$. После введения обозначения $t_n = 1$ из факта, что $P_n(t_i) = (-1)^{n-1-i}$ для $i = 0, \ldots, n$ следует, что $P_n(x)$ является чебышевским многочленом на отрезке $[t_0, 1]$, то есть он единственен и $P_n = S_n$.
	
	Из единственности многочлена без нулевого члена степени $n$, удовлетворяющего свойствам \eqref{eq:elfving:eq1} и \eqref{eq:elfving:eq2} на отрезке $[0, 1]$ следует, что оптимальный план состоящий из $n$ опорных точек сосредоточен в экстремальных точках многочлена $S_n(x)$, которые имеют вид \eqref{eq:th1:points}.
	
	Осталось показать, когда выполняется условие \eqref{eq:elfving:eq3} при $c = f'(z)$ и, соответственно, точки \eqref{eq:th1:points} являются опорными точками оптимального плана для оценки производной в точке $z$.

	Построим базисные полиномы Лагранжа степени $n$ без нулевого члена по точкам $\{x_i^*\}_{i=1}^n$
	\begin{equation*}
		L_{i}(x) = \frac{x \prod_{l=1}^n (x - x_l^*)}{x_i^8 \prod_{l \neq i}^n (x_i^* - x_l^*)}
	\end{equation*}	
	
	Так как теорема 4 о весах оптимального плана для оценки производной в случае полиномиальной модели работает для любых промежутков, то для нахождения весов можно использовать её, и тогда, соответственно, веса имеют следующий вид
	
	\begin{equation*}
		\omega_i = \frac{\abs{L_i'(z)}}{\sum_{j=1}^n \abs{L_j'(z)}}.
	\end{equation*}
	
	Введем обозначения $F = \left((x_j^*)^i\right)^n_{i, j = 1}$ и $\beta = \left( \omega_i (-1)^{i+n} \right)_{i=1}^n$. Так как $x_j^*$ при $i = 1, \ldots , n$ являются экстремальными точками многочлена $S_n$ и при этом $S_n(x_j^*) = (-1)^{j+n}$, то выполнение равенства для некоторого $h$
	\begin{equation}
		\label{th:mateq3}
		f'(z) = hF\beta
	\end{equation}
	 эквивалентно выполнению условия \eqref{eq:elfving:eq3} теоремы Элвинга для нахождения оптимального плана оценки производной в точке $z$.
	 
	Утверждение $F^{-1}F = I_n$, где $I_n$ --- единичная матрица размера $n$, а $i$--ый столбец матрицы $F$ на самом деле равен $f(x_i^*)$ ($i = 1, \ldots, n$) можно переписать, как систему равенств
	\begin{equation*}
		e_i^{\top} F^{-1} f(x_j^*) = \delta_{ij} \, , \quad i, j = 1, \ldots , n ,
	\end{equation*}
	где $\delta_{ij}$ --- дельта Кронекера, а $e_i$ --- $i$--ый единичный вектор. Поскольку в левой части равенств~\eqref{eq:toleg} содержатся многочлены без нулевого коэффициента степени не больше $n$ вычисленные в точках $x_j^*$, $j=1, \ldots , n$, а для каждого $i$ существует только одно $j$, такое, что $\delta_{ij} \neq 0$, то они определяют все базисные многочлены Лагранжа без нулевого члена степени $n$ вычисленные в точках $x_j^*$, $j=1, \ldots , n$, таким образом
	\begin{equation*}
		e_i^{\top} F^{-1} f(z) = L_i(z) \, , \quad i, j = 1, \ldots , n .
	\end{equation*}
	Если в предыдущем выражении вычислить производную по $z$ и переписать полученное выражение в векторной форме получим
	\begin{equation}
		\label{eq:vecleg}
		f'(z) = F \left( L'_1(z), \ldots, L'_n(z) \right)^\top.
	\end{equation}
	Приравняв правые части~\eqref{eq:vecleg} и~\eqref{th:mateq3} и домножив равенство на $F^{-1}$ слева, получаем, что
	\begin{equation*}
		h \beta = \left( L'_1(z), \ldots, L'_n(z) \right)^\top,
	\end{equation*}
	что с учетом введенных ранее обозначений влечет, что $\sign (L'_i(z)) = \sign((-1)^{i+n}) $, $i = 1, \ldots, n$ или, вспомнив, что экстремальным многочленом также может быть $-S(x)$,  $\sign (L'_i(z)) = \sign((-1)^{i+n+1}) $, $i = 1, \ldots, n$.
	
	Таким образом для того, чтобы доказать, что оптимальный план находится в точках $(x_i^*)_{i=1}^n$ , $i = 1, \ldots, n$ с указными ранее весами, осталось доказать равенство знаков $L'_i(z)$ и $\pm S_n(x_i^*)$. Но так как знаки экстремальных точек многочлена $S_n$ чередуются, достаточно показать при каких $z$ выражения $(-1)^i L'_i(z)$ имеет одинаковый знак для $i=1,\ldots , n$.
	
	Так как корни многочленов $L_i$ и $L_j$ для любых $i$ и $j$ таких, что $i < j$ удовлетворяют условию леммы \ref{lemma:droots}, условие которой было приведено на странице \pageref{lemma:droots}, то последовательно её применяя ко всем базисным многочленам получаем, что корни их производных, обозначения для которых были описаны в условии этой теоремы, удовлетворяют следующему соотношению:
	\begin{equation*}
		u^n_1 < u^{n-1}_1 < \ldots < u^1_1 < u^n_2 < u^{n-1}_2 < \ldots < u_{n-1}^1.
	\end{equation*}
	
	Можно видеть, что, так как все узловые точки больше нуля, знак многочлена $L_i(z)$ при $z \to -\infty$ будет равен $(-1)^{n+i+1}$. В то же время знак $L'_i(z)$ будет противоположным $L_i(z)$, так как меняется четность многочлена и при этом не меняется знак при старшем коэффициенте, то есть $\sign(L'_i(z)) = \sign((-1)^{n+i})$ при $z \to -\infty$. И, следовательно, $\sign((-1)^i L'_i(z)) = \sign((-1)^{n+2i}) = \sign((-1)^{n})$ при $z \to -\infty$, то есть $\sign((-1)^i L'_i(z))$ не зависит от $i$ и имеет постоянный знак для любых $i$, что означает, что при $z \in (-\infty, u_1^n)$ третье условие теоремы Элвинга выполняется и план является оптимальным.
	
	Осталось изучить как ведут себя знаки $\sign((-1)^i L'_i(z))$ на остальных промежутках. На промежутках $[u_j^1, u_j^n]$ ($j = 1, \ldots {n-1}$) каждый базисный многочлен меняет свой знак ровно 1 раз и на этих промежутках знаки производных не совпадают со знаками экстремального многочлена, а на промежутках $(u_j^n, u_{j+1}^1)$ ($j=1, \ldots n-2$) нет ни одного корня и поэтому $\sign((-1)^i L'_i(z)) = \sign ((-1)^{n + j}$ при $z \in (u_j^n, u_{j+1}^1)$ ($j=1, \ldots n-2$), что также подтверждает третье условие теоремы Элвинга в этих промежутках и показывает, что показанный план оптимален.
	
	На промежутке $(-\infty, u_{n-1}^n]$ каждый базисный многочлен поменял свой знак одинаковое количество раз, а так как при $z \to -\infty$ условие выполнялось, то при $z \in (u_{1, n-1}, +\infty)$ план также является оптимальным.
	
	Таким образом план эксперимента с опорными точками \eqref{eq:th1:points} и соответствующими им весами \eqref{eq:weights} является оптимальным планом для оценки производной в точке $z$ в модели без нулевого члена тогда и только тогда, когда $z \in  (-\infty, u_1^n) \cup (u_{1, n-1}, +\infty)$ или $z \in (u_{j+1}^n, u_{j}^1)$ для $j=1, \ldots n-2$. Причем в силу единственности (с точностью до знака) экстремального многочлена с $n$ экстремальными точками, не существует других планов состоящих из $n$ опорных точек, что доказывает теорему.

	\end{proof}
	
	\section{Промежуток вида $[0, d]$}
	
	В общем случае оптимальный для всех промежутков, начинающихся в нуле, существенно не отличается от случая промежутка $[0, 1]$, что будет показано в следующей теореме.
	
	\begin{theorem}
		Пусть для $i = 1, \ldots, n$ и для некоторого $d$
		\begin{equation}
			\label{eq:th2:points}
			\widehat{x_i^*} = d x_i^* ,
		\end{equation}
		где $x_i^*$ из \eqref{eq:th1:points}, и корни производной многочлена $L_i$ из формулы~\eqref{eq:lagr} построенного по точкам $\{\widehat{x_j^*}\}_j^n$ равны $u_1^i, \ldots, u_{n-1}^i$, причем $u_k^i \leqslant u_l^i$ при $k < l$. Тогда при $z \in \left(-\infty, u_1^1) \cup (u_{n-1}^n, +\infty \right)$ или $z \in (u_{j+1}^n, u_{j}^1)$ для $j=1, \ldots, n-2$ оптимальный план для нахождения производной в точке $z$ в полиномиальной модели степени $n$ без свободного члена на промежутке $[0, 1]$ имеет опорные точки $\{x_j\}_j^n$ и веса вычисленные по формуле \eqref{eq:weights}. Также при $z$ не лежащих в этих промежутках не существует оптимального плана состоящего из $n$ точек.
	\end{theorem}
	
	\begin{proof}
		Построим многочлен $S_n^d(x) = S_n(\frac{x}{d})$. Из построения ясно что его экстремальные точки равны \eqref{eq:th2:points}. Он по построению будет являться экстремальным многочленом степени $n$ на отрезке $[0, d]$ и будет повторять все свойства многочлена $S_n(x)$. Дальнейшее доказательство повторяет доказательство теоремы \ref{th1} с заменой многочлена $S_n$ на $S_n^d$.
	\end{proof}	
	
	

	
	\nocite{*}
	\bibliographystyle{gost2008}
	\bibliography{diploma}
	
\end{document}