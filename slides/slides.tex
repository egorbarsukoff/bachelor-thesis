\documentclass[unicode, notheorems, minimal, nologo]{beamer}
% If you have more than three sections or more than three subsections in at least one section,
% you might want to use the [compress] switch. In this case, only the current (sub-) section
% is displayed in the header and not the full overview.
\mode<presentation>
{
  \usetheme[numbers, totalnumbers]{Statmod}

  \setbeamercovered{transparent}
  \beamertemplatenavigationsymbolsempty
  % or whatever (possibly just delete it)
}

%\usepackage{pscyr}
\usepackage[T2A]{fontenc}
\usepackage[utf8]{inputenc}
\usepackage[russian]{babel}
\usepackage{amsthm}
\usepackage{pdfpages}
\usepackage{bibentry}

\usepackage{natbib}   % omit 'round' option if you prefer square brackets
\bibliographystyle{plainnat}

\graphicspath{{fig/}}

%\usepackage{tikz}
% you only need this when using TikZ graphics

\newtheorem{theorem}{Теорема}
\newtheorem{example}{Пример}
\newtheorem{definition}{Определение}


\DeclareMathOperator*{\E}{\mathrm{E}}
\DeclareMathOperator*{\D}{\mathrm{D}}
\DeclareMathOperator*{\sign}{sign}
\DeclareMathOperator*{\argmin}{arg\,min}
\DeclareMathOperator*{\Int}{Int}
\DeclareMathOperator*{\err}{err}
\newcommand\norm[1]{\left\lVert#1\right\rVert}
\newcommand\abs[1]{\left\lvert#1\right\rvert}


\title[Оптимальные планы для оценивания производных]{ Оптимальные планы для оценивания производных в полиномиальной регрессионной модели без свободного члена}

\author{Барсуков Егор Вячеславович, гр. 16-Б.04мм}
\institute[СПбГУ]{
    Санкт-Петербургский государственный университет \\
    Математико-Механический факультет \\
    Кафедра статистического моделирования \\
    \vspace{0.4cm}
    Научный руководитель: д.ф.-м.н., профессор В. Б. Мелас 
    \vspace{0.3cm}
}

\date{
	Санкт-Петербург \\
    2020 г.
}

\subject{Beamer}
% This is only inserted into the PDF information catalog. Can be left
% out.

% Delete this, if you do not want the table of contents to pop up at
% the beginning of each subsection:
% \AtBeginSubsection[]
% {
%   \begin{frame}<beamer>
%     \frametitle{Outline}
%     \tableofcontents[currentsection,currentsubsection]
%   \end{frame}
% }

\begin{document}


\begin{frame}
    \titlepage
\end{frame}
\begin{frame}
	\frametitle{Уравнение регрессии}
	\begin{equation*}
		y_j = \theta^\top f(x_j) + \varepsilon_j, \quad j = 1 \ldots N, \, x_j \in \mathcal{X}
	\end{equation*}
	\begin{itemize}
		\item $N$ --- количество экспериментов;
		\item $f(x)$ --- вектор регрессионных функций;
		\item $\theta = (\theta_1, \ldots, \theta_m)^\top$ --- неизвестные параметры;
		\item $x_1, \ldots, x_N$ --- условия проведения эксперимента;
		\item $\mathcal{X}$ --- множество планирования;
		\item $\varepsilon_1, \ldots, \varepsilon_N$ --- случайные величины, характеризующие ошибки наблюдений.
		\begin{itemize}
			\item Несмещенные $\mathbb{E} \varepsilon_i = 0$
			\item Некоррелированные $\mathbb{E} \varepsilon_i \varepsilon_j = 0$ для $i \neq j$
			\item Равноточные $\mathbb{E} \varepsilon_i^2 = \sigma^2$ для всех $i$
		\end{itemize}
	\end{itemize}
\end{frame}

\begin{frame}
	\frametitle{План эксперимента и информационная матрица}
	\begin{definition}
	Планом эксперимента называется следующая дискретная вероятностная мера
		\begin{equation*}
			\xi = 
				\begin{pmatrix}
				x_1 & \ldots & x_m \\
				\omega_1 & \ldots & \omega_m
			\end{pmatrix}, \quad x_i \in \mathcal{X}.
		\end{equation*}
	\end{definition}
	 
  	\begin{definition}
		Для плана эксперимента определим его информационную матрицу
		\begin{equation*}
			M(\xi) = \int_{\mathcal{X}} f(x) f^\top(x) \xi (dx).
  		\end{equation*}
	\end{definition}
  	

	
\end{frame}

\begin{frame}
	\frametitle{C-оптимальный план эксперимента}
	\begin{definition}
		C-оптимальным планом эксперимента для данного вектора $c$ называется план минимизирующий функцию $\Phi$
		\begin{equation*}
				\Phi(\xi) = \begin{cases}
				c^\top M(\xi)^{-} c, \quad \text{если } \exists v \text{, такой, что } \; c = M(\xi) v\\
				+\infty, \quad  \text{иначе}
		    \end{cases},
		\end{equation*}
		где $M(\xi)^{-}$ --- псевдообратная матрица к информационной матрице плана
	\end{definition}
	
	\begin{itemize}
		\item C-оптимальный план минимизирует дисперсию МНК-оценки~$\theta^\top c$;
		\item В общем виде задача нахождения таких планов не решена
	\end{itemize}
	
  
\end{frame}

\begin{frame}
	\frametitle{Постановка задачи}
	\begin{definition}
	Если $c = f'(z) = \left(f'_1(z), \ldots, f'_m(z) \right)^\top$ то соответствующий план называется планом для оценки производной в точке $z$.
	\end{definition}
	\begin{itemize}
		\item Целью работы является описание оптимальных планов для оценки производной в модели $f(x) = \left(x, \ldots, x^m \right)$ при носителе $\mathcal{X} = [0, d]$.
		\item Это имеет практический смысл, если существует априорное знание о значении функции в нулевой точке и эксперимент проводится на положительном отрезке.
		\item Решение задачи отличается от того, которое получается при носителе, границе которого не принадлежит ноль, которое описано в \citep{melasmain}.
		\item Разработать алгоритм численно решающий задачу нахождения c-оптимальных планов для произвольного $c$.
	\end{itemize}
	
\end{frame}

\begin{frame}
	\frametitle{Теорема Элвинга \citep{melas2010}}
	\begin{theorem}
	Допустимый план $\xi^\star$ с носителем $x_1, \ldots, x_m \in \mathcal{X}$ и весами $\omega_1, \ldots, \omega_m$ является $c$-оптимальным тогда и только тогда, когда существует $p \in \mathbb{R}^k$ и константа $h$ такие, что выполняются следующие условия:
  \begin{subequations}
  \label{eq:elfving}
  \begin{align}
	&\abs{p^\top f(x_i)} = 1 &&i=1..m \leqslant n, \\
	&\abs{p^\top f(x)} \leqslant 1  &&x \in \mathcal{X}, \\
	&c = h \sum_{i=1}^m \omega_i f(x_i) p^\top f(x_i) .
  \end{align}
  Кроме того,
  \begin{equation*}
  	h^2 = c^\top M^{-}(\xi^{*})c.
  \end{equation*}
  \end{subequations}
	\end{theorem}
\end{frame}

\begin{frame}
	\frametitle{Веса у плана для оценки производной \citep{melasmain}}
	Для набора точек $x_1^*, \ldots, x_k^*$ определим множество базисных многочленов
	\begin{equation*}
		L_i(z) = \frac{z \prod_{l \neq i} (z - x_l^*)}{x_i^* \prod_{l \neq i} (x_i^* - x_l^*)}, \, i = 1, \dots k.
	\end{equation*}
	
	\begin{theorem}
		Оптимальный план для оценивания производной полиномиальной модели без свободного члена с опорными точками $x_1^*, \ldots, x_m^*$, где $m=n$ или $m=n-1$ имеет веса вычисленные по следующей формуле:	
	\begin{equation*}
		\omega_i = \frac{\abs{L'_i(z)}}{\sum_{j=1}^m \abs{L'_j(z)}}.
	\end{equation*}
	\end{theorem}
\end{frame}

\begin{frame}
	\frametitle{Носитель плана}
	С помощью теоремы Элвинга было доказано, что если оптимальный план для оценивания производной с носителем $[0, 1]$ состоит из $n$ точек, то носитель состоит из экстремальных точек следующего многочлена
	\begin{equation*}
		S_n(x) = T_n \left(x \left(1 + \cos \frac{\pi}{2n} \right) - \cos \frac{\pi}{2n} \right),
	\end{equation*}
	где $T_n$ --- многочлен Чебышёва первого рода степени $m$. Таким образом носитель плана находится в точках
	\begin{equation*}
		x_i^* = \frac{\cos \frac{(n - i) \pi}{n} + \cos \frac{\pi}{2n}}{1 + \cos \frac{\pi}{2n}} , \, \quad i = 1, \ldots, n.
	\end{equation*}
\end{frame}

\begin{frame}
	\frametitle{Корни производных базисных многочленов}
	
	\begin{itemize}
		\item $\{L_i(z)\}_{i=1}^n$ --- базисные многочлены, построенные по точкам~$x^*_1, \ldots, x^*_n$
		\item $u_1^i, \ldots, u_{n-1}^i$ --- корни производной $i$-го базисного многочлена, упорядоченные по возрастанию
	\end{itemize}
	Было доказано, что тогда корни производных базисных многочленов упорядочены следующим образом
	\begin{equation*}
		u^n_1 < u^{n-1}_1 < \ldots < u^1_1 < u^n_2 < u^{n-1}_2 < \ldots < u_{n-1}^1.
	\end{equation*}
	
	
\end{frame}

\begin{frame}
	\frametitle{Оптимальный план на отрезке с началом в нуле для оценки производной}
	
	\begin{theorem}
		План с носителем $\{x_i^*\}_{i=1}^n$ является оптимальным планом для оценивания производной полиномиальной модели без свободного члена в точке $z$ при $\mathcal{X} = [0, 1]$ тогда и только тогда, когда выполняется одно из следующих условий:
		\begin{itemize}
			\item $z \in (-\infty, u_1^n)$
			\item $z \in (u_i^1, u_{i+1}^n), \, i = 1, \ldots, n-2$
			\item $z \in (u_{n-1}^1, +\infty)$
		\end{itemize}
	\end{theorem}
\end{frame}

\begin{frame}
	\frametitle{Пример $m = 3$}
	С помощью этой теоремы найдем планы для оценки производной в случае $f(x) = (x, x^2, x^3)^\top$, $\mathcal{X} = [0, 1]$ и покажем их оптимальность. Опорные точки плана будут совпадать с экстремальными точками многочлена $S_3(x):$
	\begin{equation*}
		S_3(x) = 4 \left(\frac{\sqrt{3}}{2}+1\right)^3 x^3-6 \sqrt{3} \left(\frac{\sqrt{3}}{2}+1\right)^2 x^2+6 \left(\frac{\sqrt{3}}{2}+1\right) x,
	\end{equation*}
	Которыми являются
	\begin{equation*}
		x_1^* = 3 \sqrt{3} - 5, \quad x_2^* = \sqrt{3} - 1 , \quad x_3^* = 1.
	\end{equation*}
	
\end{frame}

\begin{frame}
	\frametitle{Производные базисных многочленов ($m = 3$)}
	\begin{figure}
		\includegraphics[width=\textwidth]{fig/dl_color.pdf}
		\caption{Производные базисных многочленов. Красным выделены промежутки, на которых выполняется теорема.}
	\end{figure}
\end{frame}

\begin{frame}
	\frametitle{Численное нахождение оптимальных планов}
	\begin{itemize}
		\item С-оптимальность плана определяется как решение задачи оптимизации
		\item Теорема Элвинга дает критерий С-оптимальности
		\item Поэтому численный алгоритм может быть устроен следующим образом:
		\begin{enumerate}
			\item Численно оптимизируем функцию $\Phi(\xi)$ из определения со случайным начальным планом
			\item Проверяем выполнение условий теоремы Элвинга
			\begin{itemize}
				\item Если они выполнены --- у нас есть результат
				\item Если нет --- возвращаемся к п. 1
			\end{itemize}
		\end{enumerate}
	\end{itemize}
\end{frame}

\begin{frame}
	\frametitle {Оптимизация $\Phi(\xi)$}
	\begin{itemize}
		\item У любой матрицы $A^-(x)$ при условии постоянного ранга существует непрерывная вторая производная выражающаяся через производные $A(x)$
		\item Так как $\Phi(\xi) = c^\top M^-(\xi) c$, а ранг $M(\xi)$ зависит только от количества опорных точек, то существуют вторые производные у $\Phi$ по опорным точкам и весам
		\item Поэтому был использован квазиньютоновский алгоритм оптимизации с ограничениями L-BFGS-B 
	\end{itemize}
\end{frame}

\begin{frame}
	\frametitle{Проверка условий теоремы Элвинга}
	Нужно найти такой вектор $p$, что $\abs{p^\top f(x)} \leqslant 1$, при этом равенство достигается в опорных точках. Такие $p$ будут решением уравнения
	\begin{equation*}
		\begin{pmatrix}
			f_1(x_1) & \dots & f_n(x_1) \\
			f_1(x_2) & \dots & f_n(x_2) \\
  			\vdots &   \ddots & \vdots \\
  			f_1(x_n)  & \dots & f_n(x_n) \\
  			f'_1(x_{i_1}) & \dots & f'_n(x_{i_1}) \\
  			\vdots &   \ddots & \vdots \\
  			f'_1(x_{i_m})  & \dots & f'_n(x_{i_m})
		\end{pmatrix} 
		p =
		\begin{pmatrix}
			1 \\ s_2 \\ \vdots \\ s_n \\ 0 \\ \vdots \\ 0
		\end{pmatrix},
	\end{equation*}
	где $x_{i_j}$ --- опорные точки не на границе области планирования, а $s_i = \pm 1$.
\end{frame}

\begin{frame}
	\frametitle{Проверка условий теоремы Элвинга}
	Для всех $p$ с нулевой невязкой для предыдущего уравнения и данного плана $\xi$, проверяем третье условие теоремы Элвинга:
	\begin{equation*}
		c = h \sum_{i=1}^m \omega_i f(x_i) p^\top f(x_i)
	\end{equation*}
	
	\begin{itemize}
		\item Для того, чтобы не вычислять $h$ достаточно проверять коллинеарность векторов
		\item Все сравнения с точностью до машинного нуля
		\item Если равенство выполняется, то $\xi$ --- с-оптимальный план
	\end{itemize}
\end{frame}


\begin{frame}
	\frametitle{Результаты}
	\begin{itemize}
		\item Описаны оптимальные планы размера $n$ для нахождения производной в полиномиальной модели без свободного члена с $\mathcal{X} = [0, 1]$
		\item Приведен пример применения этого результата
		\item Разработан алгоритм численного нахождения с-оптимальных планов в общем случае
	\end{itemize}
\end{frame}

	\nobibliography{diploma}

\end{document}